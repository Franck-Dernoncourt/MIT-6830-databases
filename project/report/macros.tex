%%%%%%%%%% Start TeXmacs macros
% macro for todo entries
%The first argument is the name of the person who should do the specified tak. This argument is optional and defaults to ``scribe''. 
%The second argument describes the type of the task. Tasks can be 
%missing 
%expand 
%clarify 
%reference 
%incorrect 
%editing 

%The third argument contains the message describing what needs to be done. 
\definecolor{CornSilk}{rgb}{1,1,.88}
\newcommand{\todo}[3][scribe]{\textbf{\textcolor{red}{TODO: #2}} \textit{\textcolor{green}{@#1}\textcolor{blue}{ : #3}}}
%aside note: 
\newcommand{\sidenote}[1]{\fcolorbox{black}{CornSilk}{\parbox{6 in}{\textit{\textbf{\Large{Did You Know?}}}\newline #1}}}
\newcommand{\nbdiscussion}[1]{\footnote{\fcolorbox{black}{CornSilk}{\parbox{6 in}{\textit{\textbf{\Large{NB Discussion}}}\newline #1}}}}
\newcommand{\faq}[2]{\fcolorbox{black}{CornSilk}{\parbox{6 in}{\textbf{\textit{\Large{FAQ}}}\newline 
\begin{description} 
\item[\textbf{Q:}] #1
\newline 
\item[\textbf{A:}] #2
\end{description} }}}
\newcommand{\hilight}[1]{\colorbox{yellow}{#1}}
\newcommand{\tmabbr}[1]{#1}
\newcommand{\tmem}[1]{{\em #1\/}}
\newcommand{\tmop}[1]{\ensuremath{\operatorname{#1}}}
\newcommand{\tmstrong}[1]{\textbf{#1}}
\newcommand{\tmtextit}[1]{{\itshape{#1}}}
\newcommand{\tmtexttt}[1]{{\ttfamily{#1}}}
\newcommand{\getdir}{images}
\newcommand{\keyword}[1]{\index{#1|textbf}}
\newcommand{\mainword}[1]{\textbf{#1}}
\newenvironment{enumeratenumeric}{\begin{enumerate}[1.] }{\end{enumerate}}
\newenvironment{itemizedot}{\begin{itemize} \renewcommand{\labelitemi}{$\bullet$}\renewcommand{\labelitemii}{$\bullet$}\renewcommand{\labelitemiii}{$\bullet$}\renewcommand{\labelitemiv}{$\bullet$}}{\end{itemize}}
\newenvironment{tmparsep}[1]{\begingroup\setlength{\parskip}{#1}}{\endgroup}

 %%%%%%%%%% End TeXmacs macros

%set the margins to a reasonable value 
%\addtolength{\oddsidemargin}{-.75in}
%\addtolength{\evensidemargin}{-.75in}
%\addtolength{\textwidth}{1.5in}
%\addtolength{\topmargin}{-.875in}
%\addtolength{\textheight}{1.75in}

\usepackage{calc}

\makeatletter
\newcommand{\DESCRIPTION@original@item}{}
\let\DESCRIPTION@original@item\item
\newcommand*{\DESCRIPTION@envir}{DESCRIPTION}
\newlength{\DESCRIPTION@totalleftmargin}
\newlength{\DESCRIPTION@linewidth}
\newcommand{\DESCRIPTION@makelabel}[1]{\llap{#1}}%
\newcommand{\DESCRIPTION@item}[1][]{%
  \setlength{\@totalleftmargin}%
       {\DESCRIPTION@totalleftmargin+\widthof{\textbf{#1 }}-\leftmargin}%
  \setlength{\linewidth}
       {\DESCRIPTION@linewidth-\widthof{\textbf{#1 }}+\leftmargin}%
  \par\parshape \@ne \@totalleftmargin \linewidth
  \DESCRIPTION@original@item[\textbf{#1}]%
}
\newenvironment{DESCRIPTION}
  {\list{}{\setlength{\labelwidth}{0cm}%
           \let\makelabel\DESCRIPTION@makelabel}%
   \setlength{\DESCRIPTION@totalleftmargin}{\@totalleftmargin}%
   \setlength{\DESCRIPTION@linewidth}{\linewidth}%
   \renewcommand{\item}{\ifx\@currenvir\DESCRIPTION@envir
                           \expandafter\DESCRIPTION@item
                        \else
                           \expandafter\DESCRIPTION@original@item
                        \fi}}
  {\endlist}
\makeatother

% Macros for proofs, theorems, etc.
\theoremstyle{plain}% default
\newtheorem{thm}{Theorem}[section]
\newtheorem{lem}[thm]{Lemma}
\newtheorem{prop}[thm]{Proposition}
\newtheorem*{cor}{Corollary}
\newtheorem*{KL}{Klein’s Lemma}
\theoremstyle{definition}
\newtheorem{defn}{Definition}[section]
\newtheorem{conj}{Conjecture}[section]
\newtheorem{exmp}{Example}[section]
\theoremstyle{remark}
\newtheorem*{rem}{Remark}
\newtheorem*{note}{Note}
\newtheorem{case}{Case}

\newenvironment{definition}[1][Definition]{\begin{trivlist}
\item[\hskip \labelsep {\bfseries #1}]}{\end{trivlist}}
\newenvironment{example}[1][Example]{\begin{trivlist}
\item[\hskip \labelsep {\bfseries #1}]}{\end{trivlist}}
\newenvironment{remark}[1][Remark]{\begin{trivlist}
\item[\hskip \labelsep {\bfseries #1}]}{\end{trivlist}}

